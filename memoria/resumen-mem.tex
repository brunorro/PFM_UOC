\begin{center}
\Huge{Resumen}\\[2cm]
\end{center}

En esta memoria se intentará exponer todo el trabajo que se ha llevado a cabo en pos de intentar crear un sistema de reconocimiento facial con las siguientes características:
\begin{itemize}
	\item{\textbf{Basado en software libre}\\
	Las herramientas libres de que se dispone hoy día hacen este punto quizá más sencillo que el tratarlo de hacer con software no liberado.}
	\item{\textbf{Económico}\\
	Los elementos hardware necesarios para la realización del proyecto son componentes domésticos cuyo precio en el mercado es asequible. No se ha dispuesto de ningún hardware especial. }
	\item{\textbf{Extensible/sencillo de mantener}\\
	El proyectista es ingeniero informático, y ha intentado aplicar los conocimientos que obtuvo en su carrera a la realización de este proyecto.}
	\item{\textbf{Experimental}\\
	Aunque se han empleado partes correspondientes a otras investigaciones, se ha intentado huir de las alternativas existentes para el reconocimiento facial (en parte para esquivar patentes, en parte por ánimo investigador). Se ha utilizado un sistema basado en la comparación del resultado de realizar el análisis frecuencial con filtros de Gabor de ciertas regiones faciales (ojos, nariz y boca).}
	\item{\textbf{Estadísticamente probado}\\
	En la búsqueda de los límites del sistema se han extraído estadísticas para comprobar el funcionamiento del sistema.}
\end{itemize}
El esfuerzo final (implementación del software) no pudo llevarse a cabo al 100\%, si bien hay buena parte del software ya implementada en el repositorio indicado en páginas ulteriores.\\[2cm]

\begin{center}
\Large{Agradecimientos}
\end{center}

Quisiera agradecer al director de proyecto (Gregorio Robles) el apoyo prestado, así como a toda mi familia y a todos mis compañeros de la FIB (Facultad de informática de Barcelona). También quiero agradecer al lector su interés por este proyecto, la lectura de la memoria del cual espero que sea ameno y comprensible. \\

