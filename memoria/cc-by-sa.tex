% This file was converted from HTML to LaTeX with
% Tomasz Wegrzanowski's <maniek@beer.com> gnuhtml2latex program
% Version : 0.1
\section*{Código Legal de Creative Commons}
      
        \subsection*{Reconocimiento-CompartirIgual 3.0 España}
      	\begin{center}
		\includegraphics{imagenes/cc-by-sa.png}
	\end{center}

         CREATIVE COMMONS CORPORATION NO ES UN DESPACHO DE ABOGADOS Y NO PROPORCIONA SERVICIOS JURÍDICOS. LA DISTRIBUCIÓN DE ESTA LICENCIA NO CREA UNA RELACIÓN ABOGADO-CLIENTE. CREATIVE COMMONS PROPORCIONA ESTA INFORMACIÓN TAL CUAL (ON AN 'AS-IS' BASIS). CREATIVE COMMONS NO OFRECE GARANTÍA ALGUNA RESPECTO DE LA INFORMACIÓN PROPORCIONADA, NI ASUME RESPONSABILIDAD ALGUNA POR DAÑOS PRODUCIDOS A CONSECUENCIA DE SU USO.

        \subsubsection*{\emph{Licencia}}

        \par LA OBRA O LA PRESTACIÓN (SEGÚN SE DEFINEN MÁS ADELANTE) SE PROPORCIONA BAJO LOS TÉRMINOS DE ESTA LICENCIA PÚBLICA DE CREATIVE COMMONS (\textbf{\emph{CCPL}} O \textbf{\emph{LICENCIA}}). LA OBRA O LA PRESTACIÓN  SE ENCUENTRA PROTEGIDA POR LA LEY ESPAÑOLA DE PROPIEDAD INTELECTUAL Y/O CUALESQUIERA OTRAS NORMAS QUE RESULTEN DE APLICACIÓN. QUEDA PROHIBIDO CUALQUIER USO DE LA OBRA O PRESTACIÓN DIFERENTE A LO AUTORIZADO BAJO ESTA LICENCIA O LO DISPUESTO EN LA LEY DE PROPIEDAD INTELECTUAL.\\

        \par MEDIANTE EL EJERCICIO DE CUALQUIER DERECHO SOBRE LA OBRA O LA PRESTACIÓN, USTED ACEPTA Y CONSIENTE LAS LIMITACIONES Y OBLIGACIONES DE ESTA LICENCIA, SIN PERJUICIO DE LA NECESIDAD DE CONSENTIMIENTO EXPRESO EN CASO DE VIOLACIÓN PREVIA DE LOS TÉRMINOS DE LA MISMA. EL LICENCIADOR LE CONCEDE LOS DERECHOS CONTENIDOS EN ESTA LICENCIA, SIEMPRE QUE USTED ACEPTE LOS PRESENTES TÉRMINOS Y CONDICIONES. \\

        \par \textbf{1. Definiciones}

        \begin{enumerate}
          \item La \textbf{\emph{obra}} es la creación literaria, artística o científica ofrecida bajo los términos de esta licencia.

          \item En esta licencia se considera  una \textbf{\emph{prestación}} cualquier interpretación, ejecución, fonograma, grabación audiovisual, emisión o transmisión, mera fotografía u otros objetos protegidos por la legislación de propiedad intelectual vigente aplicable.

          \item La aplicación de esta licencia a una \textbf{\emph{colección}} (definida más adelante) afectará únicamente a su estructura en cuanto forma de expresión de la selección o disposición de sus contenidos, no siendo extensiva a éstos. En este caso la colección tendrá la consideración de obra a efectos de esta licencia.

          \item El \textbf{\emph{titular originario}} es: 
        \begin{enumerate}
          \item En el caso de una obra literaria, artística o científica, la persona natural o grupo de personas que creó la obra.
          \item En el caso de una obra colectiva, la persona que la edite y divulgue bajo su nombre, salvo pacto contrario.
          \item En el caso de una interpretación o ejecución, el actor, cantante, músico, o cualquier otra persona que represente, cante, lea, recite, interprete o ejecute en cualquier forma una obra. 
          \item En el caso de un fonograma, el productor fonográfico, es decir, la persona natural o jurídica bajo cuya iniciativa y responsabilidad se realiza por primera vez una fijación exclusivamente sonora de la ejecución de una obra o de otros sonidos.
          \item En el caso de una grabación audiovisual, el productor de la grabación, es decir, la persona natural o jurídica que tenga la iniciativa y asuma la responsabilidad de las fijaciones de un plano o secuencia de imágenes, con o sin sonido.
          \item En el caso de una emisión o una transmisión, la entidad de radiodifusión.
          \item En el caso de una mera fotografía, aquella persona que la haya realizado.
          \item En el caso de otros objetos protegidos por la legislación de propiedad intelectual vigente, la persona que ésta señale.
\end{enumerate}


          \item Se considerarán \textbf{\emph{obras derivadas}} aquellas obras creadas a partir de la licenciada, como por ejemplo: las traducciones y adaptaciones; las revisiones, actualizaciones y anotaciones; los compendios, resúmenes y extractos; los arreglos musicales y, en general, cualesquiera transformaciones de una obra literaria, artística o científica. Para evitar la duda, si la obra consiste en una composición musical o grabación de sonidos, la sincronización temporal de la obra con una imagen en movimiento (\emph{synching}) será considerada como una obra derivada a efectos de esta licencia.

          \item Tendrán la consideración de \textbf{\emph{colecciones}} la recopilación de obras ajenas, de datos o de otros elementos independientes como las antologías y las bases de datos que por la selección o disposición de sus contenidos constituyan creaciones intelectuales. La mera incorporación de una obra en una colección no dará lugar a una derivada a efectos de esta licencia.

          \item El \textbf{\emph{licenciador}} es la persona o la entidad que ofrece la obra o prestación bajo los términos de esta licencia y le concede los derechos de explotación de la misma conforme a lo dispuesto en ella. 

          \item \textbf{\emph{Usted}} es la persona o la entidad que ejercita los derechos concedidos mediante esta licencia y que no ha violado previamente los términos de la misma con respecto a la obra o la prestación, o que ha recibido el permiso expreso del licenciador de ejercitar los derechos concedidos mediante esta licencia a pesar de una violación anterior.

          \item La \textbf{\emph{transformación}} de una obra comprende su traducción, adaptación y cualquier otra modificación en su forma de la que se derive una obra diferente. La creación resultante de la transformación de una obra tendrá la consideración de obra derivada. 

         \item Se entiende por \textbf{\emph{reproducción}} la fijación directa o indirecta, provisional o permanente, por cualquier medio y en cualquier forma, de toda la obra o la prestación o de parte de ella, que permita su comunicación o la obtención de copias. 

	\item Se entiende por \textbf{\emph{distribución}} la puesta a disposición del público del original o de las copias de la obra o la prestación, en un soporte tangible, mediante su venta, alquiler, préstamo o de cualquier otra forma. 

     \item Se entiende por \textbf{\emph{comunicación pública}} todo acto por el cual una pluralidad de personas, que no pertenezcan al ámbito doméstico de quien la lleva a cabo,  pueda tener acceso a la obra o la prestación sin previa distribución de ejemplares a cada una de ellas. Se considera comunicación pública la puesta a disposición del público de obras o prestaciones por procedimientos alámbricos o inalámbricos, de tal forma que cualquier persona pueda acceder a ellas desde el lugar y en el momento que elija.

     \item La \textbf{\emph{explotación}} de la obra o la prestación comprende la reproducción, la distribución, la comunicación pública y, en su caso, la transformación.

     \item Los \textbf{\emph{elementos de la licencia}} son las características principales de la licencia según la selección efectuada por el licenciador e indicadas en el título de esta licencia: Reconocimiento, CompartirIgual.
\item Una \textbf{\emph{licencia equivalente}} es: 
	\begin{enumerate}
	\item Una versión posterior de esta licencia de Creative Commons con los mismos elementos de licencia.
	\item La misma versión o una versión posterior de esta licencia de cualquier otra jurisdicción reconocida por Creative Commons con los mismos elementos de la licencia (ejemplo: Reconocimiento-CompartirIgual 3.0 Japón).
        \item La misma versión o una versión posterior de la licencia de Creative Commons no adaptada a ninguna jurisdicción (\emph{Unported}) con los mismos elementos de la licencia.	
	\item Una de las licencias compatibles que aparece en http://creativecommons.org/compatiblelicenses y que ha sido aprobada por Creative Commons como esencialmente equivalente a esta licencia porque, como mínimo:
		\begin{enumerate}
	\item Contiene términos con el mismo propósito, el mismo significado y el mismo efecto que los elementos de esta licencia.
	\item Permite explícitamente que las obras derivadas de obras sujetas a ella puedan ser distribuidas mediante esta licencia, la licencia de Creative Commons no adaptada a ninguna jurisdicción (\emph{Unported}) o una licencia de cualquier otra jurisdicción reconocida por Creative Commons, con sus mismos elementos de licencia.
	        \end{enumerate}
	
        \end{enumerate}

\end{enumerate}

        \par \textbf{2. Límites de los derechos.}\\
	 Nada en esta licencia pretende reducir o restringir cualesquiera límites legales de los derechos exclusivos del titular de los derechos de propiedad intelectual de acuerdo con la Ley de propiedad intelectual o cualesquiera otras leyes aplicables, ya sean derivados de usos legítimos, tales como la copia privada o la cita, u otras limitaciones como la resultante de la primera venta de ejemplares (agotamiento).\\

        \par \textbf{3. Concesión de licencia.} \\
	Conforme a los términos y a las condiciones de esta licencia, el licenciador concede, por el plazo de protección de los derechos de propiedad intelectual y a título gratuito, una licencia de ámbito mundial no exclusiva que incluye los derechos siguientes:

        \begin{enumerate}
          \item Derecho de reproducción, distribución y comunicación pública de la obra o la prestación.
          \item Derecho a incorporar la obra o la prestación en una o más colecciones.
          \item Derecho de reproducción, distribución y comunicación pública de la obra o la prestación lícitamente incorporada en una colección.
          \item Derecho de transformación de la obra para crear una obra derivada siempre y cuando se  incluya en ésta una indicación de la transformación o modificación efectuada.
          \item Derecho de reproducción, distribución y comunicación pública de obras derivadas creadas a partir de la obra licenciada.
          \item Derecho a extraer y reutilizar la obra o la prestación de una base de datos.
	 \item Para evitar cualquier duda, el titular originario:
        \begin{enumerate}
	\item Conserva el derecho a percibir las remuneraciones o compensaciones previstas por actos de explotación de la obra o prestación, calificadas por la ley como irrenunciables e inalienables y sujetas a gestión colectiva obligatoria.
	\item Renuncia  al derecho exclusivo a percibir, tanto individualmente como mediante una entidad de gestión colectiva de derechos, cualquier remuneración derivada de actos de explotación de la obra o prestación que usted realice.
        \end{enumerate}

        \end{enumerate}

        \par Estos derechos se pueden ejercitar en todos los medios y formatos, tangibles o intangibles, conocidos en el momento de la concesión de esta licencia. Los derechos mencionados incluyen el derecho a efectuar las modificaciones que sean precisas técnicamente para el ejercicio de los derechos en otros medios y formatos. Todos los derechos no concedidos expresamente por el licenciador quedan reservados, incluyendo, a título enunciativo pero no limitativo, los derechos morales irrenunciables reconocidos por la ley aplicable. En la medida en que el licenciador ostente derechos exclusivos previstos por la ley nacional vigente que implementa la directiva europea en materia de derecho sui generis sobre bases de datos, renuncia expresamente a dichos derechos exclusivos.\\


        \par \textbf{4. Restricciones.}\\
	 La concesión de derechos que supone esta licencia se encuentra sujeta y limitada a las restricciones siguientes:

        \begin{enumerate}
          \item Usted puede reproducir, distribuir o comunicar públicamente la obra o prestación solamente bajo los términos de esta licencia y debe incluir una copia de la misma, o su Identificador Uniforme de Recurso (URI). Usted no puede ofrecer o imponer ninguna condición sobre la obra o prestación que altere o restrinja los términos de esta licencia o el ejercicio de sus derechos por parte de los concesionarios de la misma. Usted no puede sublicenciar la obra o prestación. Usted debe mantener intactos todos los avisos que se refieran a esta licencia y a la ausencia de garantías. Usted no puede reproducir, distribuir o comunicar públicamente la obra o prestación  con medidas tecnológicas que controlen el acceso o el uso de una manera contraria a los términos de esta licencia. Esta sección 4.a también afecta a la obra o prestación incorporada en una colección, pero ello no implica que ésta en su conjunto quede automáticamente o deba quedar sujeta a los términos de la misma. En el caso que le sea requerido, previa comunicación del licenciador, si usted incorpora la obra en una colección y/o crea una obra derivada, deberá quitar cualquier crédito requerido en el apartado 4.c,  en la medida de lo posible.

	\item Usted puede distribuir o comunicar públicamente una obra derivada en el sentido de esta licencia solamente bajo los términos de la misma u otra licencia equivalente. Si usted utiliza esta misma licencia debe incluir una copia o bien su URI, con cada obra derivada que usted distribuya o comunique públicamente. Usted no puede ofrecer o imponer ningún término respecto a la obra derivada que altere o restrinja los términos de esta licencia o el ejercicio de sus derechos por parte de los concesionarios de la misma. Usted debe mantener intactos todos los avisos que se refieran a esta licencia y a la ausencia de garantías cuando distribuya o comunique públicamente la obra derivada. Usted no puede ofrecer o imponer ningún término respecto de las obras derivadas o sus transformaciones que alteren o restrinjan los términos de esta licencia o el ejercicio de sus derechos por parte de los concesionarios de la misma. Usted no puede reproducir, distribuir o comunicar públicamente la obra derivada con medidas tecnológicas que controlen el acceso o uso de la obra de una manera contraria a los términos de esta licencia. Si utiliza una licencia equivalente debe cumplir con los requisitos que ésta establezca cuando distribuya o comunique públicamente la obra derivada. Todas estas condiciones se aplican a una obra derivada en tanto que incorporada a una colección, pero no implica que ésta tenga que estar sujeta a los términos de esta licencia.

          \item Si usted reproduce, distribuye o comunica públicamente la obra o la prestación, una colección que la incorpore o cualquier obra derivada, debe mantener intactos todos los avisos sobre la propiedad intelectual e indicar, de manera razonable conforme al medio o a los medios que usted esté utilizando:
            \begin{enumerate}
		\item El nombre del autor original, o el seudónimo si es el caso, así como el del titular originario, si le es facilitado.
		\item El nombre de aquellas partes (por ejemplo: institución, publicación, revista) que el titular originario y/o el licenciador designen para ser reconocidos en el aviso legal, las condiciones de uso, o de cualquier otra manera razonable.
		\item El  título de la obra o la prestación si le es facilitado.
		\item El URI, si existe, que el licenciador especifique para ser vinculado a la obra o la prestación, a menos que tal URI no se refiera al aviso legal o a la información sobre la licencia de la obra o la prestación.
		\item En el caso de una obra derivada, un aviso que identifique la transformación de la obra en la obra derivada (p. ej., 'traducción castellana de la obra de Autor Original,' o 'guión basado en obra original de Autor Original').
\end{enumerate}
Este reconocimiento debe hacerse de manera razonable. En el caso de una obra derivada o incorporación en una colección estos créditos deberán aparecer como mínimo en el mismo lugar donde se hallen los correspondientes a otros autores o titulares y de forma comparable a los mismos. Para evitar la duda, los créditos requeridos en esta sección sólo serán utilizados a efectos de atribución de la obra o la prestación en la manera especificada anteriormente. Sin un permiso previo por escrito, usted no puede afirmar ni dar a entender implícitamente ni explícitamente ninguna conexión, patrocinio o aprobación por parte del titular originario, el licenciador y/o las partes reconocidas hacia usted o hacia el uso que hace de la obra o la prestación.

          \item Para evitar cualquier duda, debe hacerse notar que las restricciones anteriores (párrafos 4.a, 4.b y 4.c) no son de aplicación a aquellas partes de la obra o la prestación objeto de esta licencia que únicamente puedan ser protegidas mediante el derecho sui generis sobre bases de datos recogido por la ley nacional vigente implementando la directiva europea de bases de datos

         
 \end{enumerate}

        \par \textbf{5. Exoneración de responsabilidad} \\

        \par A MENOS QUE SE ACUERDE MUTUAMENTE ENTRE LAS PARTES, EL LICENCIADOR OFRECE LA OBRA O LA PRESTACIÓN TAL CUAL (ON AN 'AS-IS' BASIS) Y NO CONFIERE NINGUNA GARANTÍA DE CUALQUIER TIPO RESPECTO DE LA OBRA O LA PRESTACIÓN O DE LA PRESENCIA O AUSENCIA DE ERRORES QUE PUEDAN O NO SER DESCUBIERTOS. ALGUNAS JURISDICCIONES NO PERMITEN LA EXCLUSIÓN DE TALES GARANTÍAS, POR LO QUE TAL EXCLUSIÓN PUEDE NO SER DE APLICACIÓN A USTED. \\

        \par \textbf{6. Limitación de responsabilidad.} SALVO QUE LO DISPONGA EXPRESA E IMPERATIVAMENTE LA LEY APLICABLE, EN NINGÚN CASO EL LICENCIADOR SERÁ RESPONSABLE ANTE USTED POR CUALESQUIERA DAÑOS RESULTANTES, GENERALES O ESPECIALES (INCLUIDO EL DAÑO EMERGENTE Y EL LUCRO CESANTE), FORTUITOS O CAUSALES, DIRECTOS O INDIRECTOS, PRODUCIDOS EN CONEXIÓN CON ESTA LICENCIA O EL USO DE LA OBRA O LA PRESTACIÓN, INCLUSO SI EL LICENCIADOR HUBIERA SIDO INFORMADO DE LA POSIBILIDAD DE TALES DAÑOS. \\


        \par \textbf{7. Finalización de la licencia} 

        \begin{enumerate}
          \item Esta licencia y la concesión de los derechos que contiene terminarán automáticamente en caso de cualquier incumplimiento de los términos de la misma. Las personas o entidades que hayan recibido de usted obras derivadas o colecciones bajo esta licencia, sin embargo, no verán sus licencias finalizadas, siempre que tales personas o entidades se mantengan en el cumplimiento íntegro de esta licencia. Las secciones 1, 2, 5, 6, 7 y 8 permanecerán vigentes pese a cualquier finalización de esta licencia.

          \item Conforme a las condiciones y términos anteriores, la concesión de derechos de esta licencia es vigente por todo el plazo de protección de los derechos de propiedad intelectual según la ley aplicable. A pesar de lo anterior, el licenciador se reserva el derecho a divulgar o publicar la obra o la prestación en condiciones distintas a las presentes, o de retirar la obra o la prestación en cualquier momento. No obstante, ello no supondrá dar por concluida esta licencia (o cualquier otra licencia que haya sido concedida, o sea necesario ser concedida, bajo los términos de esta licencia), que continuará vigente y con efectos completos a no ser que haya finalizado conforme a lo establecido anteriormente, sin perjuicio del derecho moral de arrepentimiento en los términos reconocidos por la ley de propiedad intelectual aplicable.
        \end{enumerate}

        \par \textbf{8. Miscelánea}

        \begin{enumerate}
          \item Cada vez que usted realice cualquier tipo de explotación de la obra o la prestación, o de una colección que la incorpore, el licenciador ofrece a los terceros y sucesivos licenciatarios la concesión de derechos sobre la obra o la prestación en las mismas condiciones y términos que la licencia concedida a usted.

          \item Cada vez que usted realice cualquier tipo de explotación de una obra derivada, el licenciador  ofrece a los terceros y sucesivos licenciatarios la concesión de derechos sobre la obra objeto de esta licencia en las mismas condiciones y términos que la licencia concedida a usted. 

          \item Si alguna disposición de esta licencia resulta inválida o inaplicable según la Ley vigente, ello no afectará la validez o aplicabilidad del resto de los términos de esta licencia y, sin ninguna acción adicional por cualquiera las partes de este acuerdo, tal disposición se entenderá reformada en lo estrictamente necesario para hacer que tal disposición sea válida y ejecutiva.

          \item No se entenderá que existe renuncia respecto de algún término o disposición de esta licencia, ni que se consiente violación alguna de la misma, a menos que tal renuncia o consentimiento figure por escrito y lleve la firma de la parte que renuncie o consienta. 

          \item Esta licencia constituye el acuerdo pleno entre las partes con respecto a la obra o la prestación objeto de la licencia. No caben interpretaciones, acuerdos o condiciones  con respecto a la obra o la prestación que no se encuentren expresamente especificados en la presente licencia. El licenciador no estará obligado por ninguna disposición complementaria que pueda aparecer en cualquier comunicación que le haga llegar usted. Esta licencia no se puede modificar sin el mutuo acuerdo por escrito entre el licenciador y usted.
        \end{enumerate}
        
          \subsubsection*{Aviso de Creative Commons}

          \par Creative Commons no es parte de esta licencia, y no ofrece ninguna garantía en relación con la obra o la prestación. Creative Commons no será responsable frente a usted o a cualquier parte, por cualesquiera daños resultantes, incluyendo, pero no limitado, daños generales o especiales (incluido el daño emergente y el lucro cesante), fortuitos o causales, en conexión con esta licencia. A pesar de las dos (2) oraciones anteriores, si Creative Commons se ha identificado expresamente como el licenciador, tendrá todos los derechos y obligaciones del licenciador. \\

          \par Salvo para el propósito limitado de indicar al público que la obra o la prestación está licenciada bajo la CCPL, ninguna parte utilizará la marca registrada 'Creative Commons' o cualquier marca registrada o insignia relacionada con 'Creative Commons' sin su consentimiento por escrito. Cualquier uso permitido se hará de conformidad con las pautas vigentes en cada momento sobre el uso de la marca registrada por 'Creative Commons', en tanto que sean publicadas su sitio web (website) o sean proporcionadas a petición previa. Para evitar cualquier duda, estas restricciones en el uso de la marca no forman parte de esta licencia. \\

          \par Puede contactar con Creative Commons en: http://creativecommons.org/.
        
      
    

